\documentclass[a4paper,12pt]{article}
\begin{document}
\section*{SICP Exercise 1.9}

\subsection*{Question}
There is a clever algorithm for computing the Fibonacci numbers in a logarithmic 
number of steps. Recall the transformation of the state variables $a$ and $b$ in the fib-iter process of 
section 1.2.2: $a \leftarrow a + b$ and $b \leftarrow a$. Call this transformation $T$, and observe that applying $T$ over and over 
again $n$ times, starting with $1$ and $0$, produces the pair $Fib(n + 1)$ and $Fib(n)$. In other words, the 
Fibonacci numbers are produced by applying $T^n$, the $n$th power of the transformation $T$, starting with the 
pair $(1,0)$. Now consider $T$ to be the special case of $p = 0$ and $q = 1$ in a family of transformations $T_{pq}$, 
where $T_{pq}$ transforms the pair $(a,b)$ according to $a \leftarrow bq + aq + ap$ and $b \leftarrow bp + aq$. Show that if we 
apply such a transformation $T_pq$ twice, the effect is the same as using a single transformation $T_{p'q'}$ of the 
same form, and compute $p'$ and $q'$ in terms of $p$ and $q$. This gives us an explicit way to square these 
transformations, and thus we can compute $T^n$ using successive squaring, as in the fast-expt 
procedure.

\subsection*{Answer}

We have $T_{pq}$ defined as:
\begin{eqnarray*}
a' & = & bq + aq + ap \\
b' & = & bp + aq
\end{eqnarray*}

We are asked to find $p'$ and $q'$ such that $T_{p'q'} = T^2_{pq} $.  In other words we want to find $p'$ and $q'$ such that:
\begin{eqnarray}
a'' & = & bq' + aq' + ap' \\
b'' & = & bp' + aq'
\end{eqnarray}

To do this we start by expanding $T_{pq}$ into itself as if we are solving for $a''$ and $b''$:
\begin{eqnarray*}
a'' & = & b'q + a'q + a'p \\
a'' & = & (bp + aq)q + (bq + aq + ap)q + (bq + aq + ap)p \\
b'' & = & b'p + a'q \\
b'' & = & (bp + aq)p + (bq + aq + ap)q
\end{eqnarray*}

The goal then is to manipulate $a''$ and $b''$ to be in the form of equations 1 and 2 so we can solve for $p'$ and $q'$.  Starting with the easier of the two equations ($b''$):
\begin{eqnarray*}
b'' & = & (bp + aq)p + (bq + aq + ap)q \\
b'' & = & bp^2 + apq + bq^2 + aq^2 + apq \\
b'' & = & b(p^2 + q^2) + a(2pq + q^2) \\
{therefore} \\
p'  & = & p^2 + q^2 \\
q'  & = & 2pq + q^2
\end{eqnarray*}

We do the same for $a''$ to ensure the discoverd $p'$ and $q'$ solve it as well:
\begin{eqnarray*}
a'' & = & b'q + a'q + a'p \\
a'' & = & (bp + aq)q + (bq + aq + ap)q + (bq + aq + ap)p \\
a'' & = & bpq + aq^2 + bq^2 + aq^2 + apq + bpq + apq + ap^2 \\
a'' & = & 2bpq + aq^2 + bq^2 + aq^2 + 2apq + ap^2 \\
a'' & = & b(2pq + q^2) + a(2pq + q^2) + a(p^2 + q^2)
\end{eqnarray*}

Both $a''$ and $b''$ were solvable with the same $p'$ and $q'$ so we have our answer to plug into the given procedure:
\begin{eqnarray*}
p'  & = & p^2 + q^2 \\
q'  & = & 2pq + q^2
\end{eqnarray*}

\end{document}



